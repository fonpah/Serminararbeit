Users of the internet have for over a decade needed a way to securely visit data(mail, websites, servers, etc.). Nonetheless, most frequently accessed data sources cannot for instance actively push their crypto certificates to all of their users (websites like Google would need to exactly know ahead of time who will visit them). Furthermore, pushing a crypto key over the same communication channel that could be a subject to a Man in the Middle (MitM) attack,could cause key learning  vulnerable too. Still internet users require a method to ''securely'' learn crypto certificates for all data sources that they will ever(inlcuding those they may never visit), and acertain if these certifcate holders(named entities) are ''trustworthy'' (that means, a named entity  may still act maliciouly even though it has a valid certificate).\\
\indent Nowaredays, we depend on certain organizations called Certificate Authority(CAs) to do this task  for us. That is, these are organisations the Relyign Parties can trust to confirm both ''\italic{authenticity}'' and ''\italic{trustworthiness}'' of all the named entities' certificates. Each of these CAs  is tied to a certificated of its own. Large software vendors like Apple, Mozilla Foundation, Microsoft, etc often configure  the X.509 certificates for approximately 160 CAs for internet users when their products are installed. These root certificates form the base-line for certificate validation by softwares like web browsers. Here, the reponsibility to choose and update the CA lists is done by the software vendors, and the verification of all data sources that a user may ever visit is done by these root CAs. However the CA Model has some ''weaknesses'' which make it vulnerable to serveral  attacks which are both difficult to detect and/or easy to implement. More details about these liaiblities will be discussed in Section \Rmnum{4} and specific examples in Section \Rmnum{5}\\
\indent The DNS Security Extensions(DNSSEC)[6], [7], [8] has recently become an operationally important technology, and their usage has been growing constantly  for six years now[5]. It has provided the a means for domain owners  to directly manage their security in same distributed database that internet users trust to find their service(DNS). The DNS-Based Authentication of Named Entities(DANE) offers the possibility to use DNSSEC to validate TLS Keys and certificates used by HTTPS and other TLS-based protocols[1]. Furthermore, a couple of commercial products such as Firefox have  add-ons for most browsers while others like Google's Chrome [10] have integrated a native support for the DANE. A remarkable advantage with the DANE approach is that it uses an already existing infrastructure(DNS), which has been used for online transactions and which a vast majority of internet client acting as relying parties(RP) are alread addicted to,(for example it is rare to find a URL that is made up of IP address) attest certificates. Thereby decreasing the system's dependencies on addtional systems  and protocols and consequently reducing the overall attack chances.\\
\indent Perspectives on the other hand tries to prevent these attack by using a collection of ''notary'' servesr that observes named data source's public key through serveral network vantage point and keep records fo a server's key over time . Clients can download these records, when needed and compare them against unauthentuicated keys, there by preventing attacks. Key observations  gathered over the multiple vantage points, makes it difficult for an attacker to compromise all the network paths to destination ,notary data that will allow the client to discover that an attack is eminent(\italic{spial redundancy}) [0]. Futhermore, it enable clients to identify malicious notaries that supply inconsistent data,and there reducing the damage of attack on the notary infrastructure(\italic{data redundancy}).\\
\indent Name Constraints is an extension of X.509 Version 3 certifcate. It defines a name space in which all subsequent certificates in the certification path \textbf{MUST} be located [rfc]. That a top-level domain can only validate certificate of subdomain in its namespace. In case of an attack on a top-level domain(its cerificate is compromised) only certifacates of sub-domains in that namespace will be affected, therey reduicing the attack surface. Details on the above mentioned alternative approaches will be discussed in Section \Rmnum{6}.    
